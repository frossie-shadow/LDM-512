\subsubsection{Use of JIRA as a risk management tool}

{\color{red} Tim -- this is a section for you to update}


Jira is an issue tracking tool that has been adopted by LSST to assist and coordinate a software development effort distributed throughout Europe. This section describes how Jira will be used as a risk management tool.

In particular, for a risk reported as a JIRA issue:
\begin{itemize}
\item The automatically generated JIRA issue number will be used as a unique alphanumeric risk identifier.
\item The JIRA Summary serves as the short descriptive title of the risk.
\item The JIRA Description serves as the full description of the risk.
\item Under the JIRA Risk Register project, additional customised fields to enter the risk likelihood, severity. Changes to these fields are automatically logged by JIRA.
\item Recommendations by the RMT are reported in the custom fields "Recommendations" and "Actions".
\end{itemize}


\subsubsection{Risk submission in JIRA}

 When submitting a new risk in JIRA a set of fields are to be fulfilled, some are optional.
{\color{red} THIS NEEDS REVIEW}
 The table below present these fields:
\begin{longtable}{|l|p{0.6\textwidth}|}\hline
{\bf JIRA field} & {\bf Description} \\ \hline
Summary    & Summary of the risk  \\ \hline
Description    & Complete description:
\begin{itemize}
  \item Detail of the risk.
  \item Cause.
  \item Possible action.
\end{itemize}
\\
\hline
Consequence  & Description of the consequences in terms of time, resource, performance.\\
\hline
Risk severity (optional)   & See \ref{sect:scoring}. \\
\hline
Risk likelihood (optional)   & See \ref{sect:scoring}. \\
\hline
\end{longtable}

Note that any new risk with a vague description will be \textbf{closed} by the RMT.
After review by the RMT, each registered risk shall have at the following fixed properties :

\begin{itemize}
\item A unique alphanumerical identifier that, once assigned, will not change.  It is solely the task of the RMT to assign this identifier.
\item A short descriptive title.
\item A full description (including both cause and proposed solution).
\item Consequence of the risk in terms of time, resource, performance.
\item The name of the original submitter.
\item The date of submission.
\item Estimated date of occurrence
\end{itemize}

In addition, each registered risk shall have the following transient properties (i.e. properties that change with time) also assigned by the RMT:
\begin{itemize}
\item A severity and likelihood rating, and the derived the risk index.
\item The date of assessment (first assessment coincides with the date of risk registration).
\item Recommended response at time of assessment (reduction or acceptance).
\item JIRA status: assigned, resolved or closed.
\item Actions taken (if any), when and by whom.
\item \ldots
\end{itemize}

The submitter may provide an initial risk rating; in this case these ratings must be re-evaluated by the RMT at time of registration.

The risk identifier may contain information of the source of submission (for example, a CU number), but {\bf not} the risk assessment scores, as these will change with time.

The list of registered risks at the DPACE level constitutes the DPAC Risk Register. The DPAC Risk Register shall be reissued after each risk management meeting.

\subsubsection{Risk Process Vs JIRA Status \label{sect:JIRA}}


The figure below presents the different JIRA status through the Risk Process, the blue boxes are the JIRA status:

\begin{figure}[H]
\begin{center}
%\includegraphics[scale=0.9]{RiskJIRAStatus}
\end{center}
\caption{JIRA Status through Risk Process}
\label{fig:riskJIRA}
\end{figure}


Note that all 'unacceptable' risk with an 'assigned' JIRA Status have at least one dedicated action monitored in JIRA.

\subsubsection{Risk change tracking}

An identified risk has a natural "life-cycle": a beginning, a middle and an end. These three parts correspond to risk registration, a monitoring phase and eventual risk closing. All risks surviving to the end of project are of course retired at project's end. During its lifespan a risk is constantly monitored and reassessed, perhaps acted on and (perhaps) reduced. In short, a risk accumulates a history.

The history of a risk is essentially a history of all past values of its transient properties. Tracking the history of risks is an important part of risk monitoring.

In addition to the Risk Register, the RMT shall issue a Risk Report from JIRA after each risk management meeting, with summary statistics of the current risks and overall risk trends.

